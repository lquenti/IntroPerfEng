\section{Introduction}
[Any page limit starts with the main content here and ends with the biography (before the appendix).

The introduction is the most important part of such a report. Its general structure is similar to the Abstract but with about one paragraph per section as listed again below.
\begin{enumerate}
\item  A general statement introducing the broad research area of the particular topic being investigated.
\item  An explanation of the specific problem (difficulty, obstacle, challenge) to be solved.
\item  A review of existing or standard solutions to this problem and their limitations.
\item  An outline of the proposed new solution.
\item  A summary of how the solution was evaluated and what the outcomes of the evaluation were.
\end{enumerate}
Furthermore, the introduction should have a contributions section, that summarizes, possibly as a list, what the report and the work described in the report has contributed.
Finally, the introduction should end with an outline of the remaining report.

\ac{HPC} refers to the usage of powerful compute systems to solve non-trivial problems.

\subsection{Citation and figure example}
Hawthorn et al. \cite{Reference1} talk about laser, which is similar to the work of his colleagues \cite{Reference2, Reference3}.
\begin{figure}[th]
\centering
\includegraphics[width=0.7\textwidth]{Electron}
\caption[An Electron]{An electron (artist's impression).}
\label{fig:Electron}
\end{figure}

\subsection{Table and listing example}
\Cref{tab:treatments} shows an example for a table in \LaTeX.

\begin{table}[th]
\caption{The effects of treatments X and Y on the four groups studied.}
\label{tab:treatments}
\centering
\begin{NiceTabular}{lrr}
\CodeBefore
\rowcolors{2}{gray!10}{white}
\Body
\textbf{Groups} & \textbf{Treatment X} & \textbf{Treatment Y} \\
1 & 0.20 & 0.80\\
2 & 0.17 & 0.70\\
3 & 0.24 & 0.75\\
4 & 0.68 & 0.30\\
\end{NiceTabular}
\end{table}

[If you show results in tables, always ensure all of them are aligned to the right and have the same precision.]

\Cref{lst:hello} shows an example of a listing, a good way to display code in \LaTeX.

\begin{listing}[th]
	\begin{minted}{Go}
package main
import "fmt"
func main() {
    fmt.Println("Hello, world!")
}
	\end{minted}
\caption{"Hello, world!" in Go}
\label{lst:hello}
\end{listing}

\section{Conclusion}
\lipsum[2]
